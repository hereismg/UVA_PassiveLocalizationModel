\begin{center}
    \textbf{\fontsize{20}{1.5}无人机遂行编队飞行中纯方位无源定位研究}

    \textbf{摘 要}
\end{center}

纯方位无源定位是编排无人机在某种场景下工作的一种有效编排方法,依题设可知无人机群的编排图形是一个同高度的某个图形,故该无人机群遂行飞行编排是否为所需求的图形形状是整个定位模型的关键。本文通过建立基于目标定位的调节模型,求解出了能够满足无人机群在题设要求条件下的定位模型,并对后续略有偏差的无人机及无人机群编排图形进行了研究,提出了相应的调整方案。

\textbf{针对问题1(1)}:为了建立定位模型,本文运用数学几何求解。首先在空间直角坐标系中经过初步建立有关几何关系的定位模型,已知该十架无人机处于同一高度的平面圆上,然而除其中三架无人机外其余无人机均略有偏差。因此引入偏差 来调节整个定位模型来求解函数,通过解析几何的算法得到了理想定位模型(见公式(1-21))。最后对定位模型进行了数据模拟检验,通过分析函数模型在 呈现的直观图可得该定位模型符合求解。

\textbf{针对问题1(2)}:为了实现无人机的有效定位,本文提出了三圆定点模型。首先通过分析已知条件的几何关系:编号00与01无人机为发射信号,且以编号00无人机为圆心,编号00与01无人机距离为半径的圆W。若要使最后的遂行飞行编队为圆形,则可假设定义第三架无人机与编号00无人机连线画圆可两圆交于一点,此时无法确定第三架无人机是否无偏差在圆W上,再次假设定义第四架无人机与编号00无人机连线画圆,可得出两点重合,即为三圆定点。最终得出还需2架无人机发射信号才能实现无人机有效定位。

\textbf{针对问题1(3)}:为了给出具体的调整方案,本文通过已知条件在问题1(2)有效定位模型的基础上,对表1数据分析可得,编号00与01的无人机是无偏差的,即最终无人机群是以半径为100m的圆形。由问题1(2)的分析结果可得,再以距离理想圆周上偏差最小的随机选择两架无人机进行有效定位,故此调整方案基于问题1(2)的有效定位模型重复上述操作。(后续正确位置在附录中显示)

\textbf{针对问题2}:为了设计锥形无人机编队的调整方案,由已知条件可知,15架无人机组成锥形编队队形,且直线上相邻两架无人机间距相等,即每3架组成一个等边三角形进行遂行飞行。基于问题1(3)的分析调整方案可得:至少要有5架无人机为无偏差编排,且由右向左依次为每一列至少有一架无人机是无偏差编排,故由此重复问题1(3)的有效定位模型。

在文章的最后,本文客观的指出了模型的优缺点,并提出了本文建立模型的算法经过修改可应用到解决日常无人机用以编排无人机群技术展览等商业领域。

\textbf{关键词}:纯方位无源定位 目标定位 定位模型 数据模拟检验 调整方案
